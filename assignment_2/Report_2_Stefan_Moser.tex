\documentclass{paper}

%\usepackage{times}
\usepackage{epsfig}
\usepackage{graphicx}
\usepackage{mathtools}
\usepackage{amssymb}
\usepackage{color}
\usepackage{caption}
\usepackage{subcaption}

% load package with ``framed'' and ``numbered'' option.
%\usepackage[framed,numbered,autolinebreaks,useliterate]{mcode}

% something NOT relevant to the usage of the package.
\setlength{\parindent}{0pt}
\setlength{\parskip}{18pt}






\usepackage[latin1]{inputenc} 
\usepackage[T1]{fontenc} 

\usepackage{listings} 
\lstset{% 
   language=Matlab, 
   basicstyle=\small\ttfamily, 
} 



\title{Report for assignment 2}



\author{Moser Stefan\\09-277-013}
% //////////////////////////////////////////////////


\begin{document}



\maketitle

\section{Binocular Stereo (Due on 18/11/2013)}

In this assignment, we retrieve the fundamental matrix of an image pair by using the normalized 8-Point-Algorithm, as described by Richard Hartley \cite{601246}. We show our results by drawing epipolar lines and the epipole for arbitrary points on various scenes. Further, we estimate the essential matrix given the intrinsic parameters of a camera. yadayada

\subsection{Epipolar line estimation}

\subsubsection{Normalizing points}

First, the user input is turned into homogeneous coordinates by appending a 1 to every point. 
In a further step, the centroid is removed from each point and the distance to the center normalized to two pixels. For this purpose, we construct a normalization matrix $M_l$ for the given points $l$, their centroid $c$ and their mean distance to the centroid $\overline{d}_l$ as:
\begin{equation}
M_l =
\begin{pmatrix}
	2/\overline{d}_l & 0 & -2/\overline{d}_x c_x \\
	0 & 2/\overline{d}_l & -2/\overline{d}_x c_y \\
	0 & 0 & 1
\end{pmatrix}
\end{equation}
So when multiplying it with an arbitrary point $(x,y,1)^T$ we get
\begin{equation}
M_l \begin{pmatrix}
 x \\
 y \\
 1
\end{pmatrix}
 = \begin{pmatrix}
 	\frac{2x}{\overline{d}_l} - \frac{2}{\overline{d}_l}c_x \\
 	\frac{2y}{\overline{d}_l} - \frac{2}{\overline{d}_l}c_y \\
 	1
 \end{pmatrix}
\end{equation}
which has exactly the desired effect described above. 
\subsection{Model reconstruction}
\bibliographystyle{plain}
\bibliography{Report_2_Stefan_Moser}
\end{document}