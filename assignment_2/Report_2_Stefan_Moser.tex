\documentclass{paper}

%\usepackage{times}
\usepackage{epsfig}
\usepackage{graphicx}
\usepackage{mathtools}
\usepackage{amssymb}
\usepackage{color}
\usepackage{caption}
\usepackage{subcaption}

% load package with ``framed'' and ``numbered'' option.
%\usepackage[framed,numbered,autolinebreaks,useliterate]{mcode}

% something NOT relevant to the usage of the package.
\setlength{\parindent}{0pt}
\setlength{\parskip}{18pt}






\usepackage[latin1]{inputenc} 
\usepackage[T1]{fontenc} 

\usepackage{listings} 
\lstset{% 
   language=Matlab, 
   basicstyle=\small\ttfamily, 
} 



\title{Report for assignment 2}



\author{Moser Stefan\\09-277-013}
% //////////////////////////////////////////////////


\begin{document}



\maketitle

\section{Binocular Stereo (Due on 18/11/2013)}

In this assignment, we retrieve the fundamental matrix of an image pair by using the normalized 8-Point-Algorithm, as described by Richard Hartley \cite{601246}. We show our results by drawing epipolar lines and the epipole for arbitrary points on various scenes. Further, we estimate the essential matrix given the intrinsic parameters of a camera. yadayada

\subsection{Epipolar line estimation}

\subsubsection{Normalizing points}

First, the user input is turned into homogeneous coordinates by appending a 1 to every point. 
In a further step, the centroid is removed from each point and the distance to the center normalized to two pixels. For this purpose, we construct a normalization matrix $M_p$ for the given points $p$, their centroid $c$ and their mean distance to the centroid $\overline{d}$ as:
\begin{equation}
M_p =
\begin{pmatrix}
	2/\overline{d} & 0 & -2/\overline{d} \cdot c_x \\
	0 & 2/\overline{d} & -2/\overline{d} \cdot c_y \\
	0 & 0 & 1
\end{pmatrix}
\end{equation}
So when multiplying it with an arbitrary point $(x,y,1)^T$ we get
\begin{equation}
M_p \begin{pmatrix}
 x \\
 y \\
 1
\end{pmatrix}
 = \begin{pmatrix}
 	\frac{2x}{\overline{d}_x} - \frac{2}{\overline{d}_x}c_x \\
 	\frac{2y}{\overline{d}_y} - \frac{2}{\overline{d}_y}c_y \\
 	1
 \end{pmatrix}
\end{equation}
which has exactly the desired effect described above. Points normalized in such 
a way will be designated as $\hat{l}$

\subsubsection{Computing Fundamental matrix}
For the given, normalized corresponding points from the base image (or left image in our case) $p$ and from the second (or right) image $p'$ 
we want to solve 
\begin{equation}
	p'^T F p = 0
\end{equation}
which is, as shown on the slides, the same as minimizing
\begin{equation}
	\sum_i ({p'}_i^T F p_i)^2
\end{equation}
under the constraint $||F||^2 = 1$. Minimizing this sum can be done using the SVD of a matrix $K$ with rows $i$ as
\begin{equation}
K_i = 
\begin{pmatrix}
 u_i'u_i & u_i'v_i & u_i' & v_i'u_i & v_i'v_i & v_i' & u_i & v_i & 1 
\end{pmatrix}
\end{equation}
with $p_i = (u_i,v_i,1)$, $p_i' = (u_i',v_i',1)$ for every pair of corresponding points. When we now compute the SVD
\begin{equation}
	[U_K, D_K, V_K] = \text{svd}(K)
\end{equation}
From the construction of the SVD we know, that the eigenvector corresponding
 to the minimal eigenvalue (ie the last column of $V_K$) does meet all
  requirements mentioned above. So it corresponds to our normalized
   fundamental matrix $\hat{F}_{r3}$. To ensure rank 2, we use again SVD, 
   discard the third (and smallest) eigenvalue and reconstruct $\hat{F}_{r2}$. 
To obtain our final fundamental matrix $F$, we remove the normalization introduced in the previous section by computing 
   \begin{equation}
   F = M_{p'} \hat{F}_{r2} M_p
   \end{equation}
\subsection{Model reconstruction}
\bibliographystyle{plain}
\bibliography{Report_2_Stefan_Moser}
\end{document}