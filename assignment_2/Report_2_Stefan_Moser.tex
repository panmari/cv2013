\documentclass{paper}

%\usepackage{times}
\usepackage{epsfig}
\usepackage{graphicx}
\usepackage{mathtools}
\usepackage{amssymb}
\usepackage{color}
\usepackage{caption}
\usepackage{subcaption}
\usepackage{natbib}

% load package with ``framed'' and ``numbered'' option.
%\usepackage[framed,numbered,autolinebreaks,useliterate]{mcode}

% something NOT relevant to the usage of the package.
\setlength{\parindent}{0pt}
\setlength{\parskip}{18pt}






\usepackage[latin1]{inputenc} 
\usepackage[T1]{fontenc} 

\usepackage{listings} 
\lstset{% 
   language=Matlab, 
   basicstyle=\small\ttfamily, 
} 



\title{Report for assignment 2}



\author{Moser Stefan\\09-277-013}
% //////////////////////////////////////////////////


\begin{document}



\maketitle

\section{Binocular Stereo (Due on 18/11/2013)}

In this assignment, we take a closer look at the
 the normalized 8-Point-Algorithm, as described by Richard Hartley \cite{601246}.

\subsection{Epipolar line estimation}

We let the user choose at least 8 corresponding points, then use this information to compute the fundamental matrix and draw epipolar lines.

\subsubsection{Normalizing points}

First, the user input is turned into homogeneous coordinates by appending a 1 to every point. 
In a further step, the centroid is removed from each point and the distance to the center normalized to two pixels. For this purpose, we construct a normalization matrix $M_p$ for the given points $p$, their centroid $c$ and their mean distance to the centroid $\overline{d}$ as:
\begin{equation}
M_p =
\begin{pmatrix}
	2/\overline{d} & 0 & -2/\overline{d} \cdot c_x \\
	0 & 2/\overline{d} & -2/\overline{d} \cdot c_y \\
	0 & 0 & 1
\end{pmatrix}
\end{equation}
So when multiplying it with an arbitrary point $(x,y,1)^T$ we get
\begin{equation}
M_p \begin{pmatrix}
 x \\
 y \\
 1
\end{pmatrix}
 = \begin{pmatrix}
 	\frac{2x}{\overline{d}} - \frac{2}{\overline{d}}c_x \\
 	\frac{2y}{\overline{d}} - \frac{2}{\overline{d}}c_y \\
 	1
 \end{pmatrix}
\end{equation}
which has exactly the desired effect described above.

\subsubsection{Computing Fundamental matrix}
For the given, normalized corresponding points from the base image (or left image in our case) $p$ and from the second (or right) image $p'$ 
we want to solve 
\begin{equation}
	p'^T F p = 0
\end{equation}
which is, as shown on the slides, the same as minimizing
\begin{equation}
	\sum_i ({p'}_i^T F p_i)^2
\end{equation}
under the constraint $||F||^2 = 1$. Minimizing this sum can be done using the SVD of a matrix $K$ with rows $i$ as
\begin{equation}
K_i = 
\begin{pmatrix}
 u_i'u_i & u_i'v_i & u_i' & v_i'u_i & v_i'v_i & v_i' & u_i & v_i & 1 
\end{pmatrix}
\end{equation}
with $p_i = (u_i,v_i,1)$, $p_i' = (u_i',v_i',1)$ for every pair of corresponding points $i$. When we now compute the SVD
\begin{equation}
	[U_K, D_K, V_K] = \text{svd}(K)
\end{equation}
From the construction of the SVD we know, that the column of $V_K$ corresponding to the minimal eigenvalue (ie. the last) does meet all
requirements mentioned above. So it corresponds to our normalized
fundamental matrix $\hat{F}_{r3}$. To ensure rank 2 we use again SVD, 
discard the third (and smallest) eigenvalue and reconstruct $\hat{F}_{r2}$. 
To obtain our final fundamental matrix $F$, we remove the normalization introduced in the previous section by computing 
   \begin{equation}
   F = M_{p'}^ T \hat{F}_{r2} M_p
   \end{equation}
   
\subsubsection{Drawing epipolar lines}
To draw the epipolar line $e$ corresponding to the point $p$ on the base image, we simply compute
\begin{equation}
	e = F p
\end{equation}
$e$ can then be used to extrapolate image coordinates, through which a line then can be matched. 
To cover every edge case, I extrapolated four points: $x_0 = 0, y_1 = 0, x_2 = w, y_3 = h$ (with $h$ as image height and $w$ as width). 
In the same way, the epipolar line for a given point $p'$ of the other image can be computed as
\begin{equation}
	e' = F^Tp'
\end{equation} 

\subsubsection{Computing the epipoles}
We know that the epipoles lie in the null space of $F$ respectively $F'$. For
obtaining them, we can simply use the matlab function \emph{null} and scale 
the result to image coordinates by scaling it in such a way that 
the z-coordinate is one.
 
\subsection{Model reconstruction}
This time, the corresponding points are not given as user input, but
read out from a file. We take an arbitrary large subset of them 
(larger than 8) to compute the fundamental
matrix and further the essential matrix. Then we estimate the depth of these points.

\subsubsection{Computing the essential matrix}
We retrieve the fundamental matrix $F$ as described in the section before. To
retrieve the essential matrix, we need to use the cameras intrinsic parameters $K$ respectively $K'$ for the secondary view:
\begin{align}
	 F &= K'^{-T} E K^{-1} \\
	\Longleftrightarrow E &= K'^T F K
\end{align}
For our computation, we had $K = K'$ given. 
\subsubsection{Estimate the rotation and translation}

From the slides we know, that the essential matrix consists of a translation $T_\times$ (expressed as skew symmetric matrix) and a rotation $R$:
\begin{equation}
	E = T_\times R
\end{equation}
To split E into these two parts, we can use the SVD of $E$, which gives us
\begin{equation}
	E = UDV^T
\end{equation}
If we define now
\begin{equation}
 W = \begin{pmatrix}
 	0 & -1 & 0 \\
 	1 & 0 & 0 \\
 	0 & 0 & 1
 \end{pmatrix}
\end{equation}
So we can easily derive the translation
\begin{equation}
T_\times = V W D V^T
\end{equation}
and the rotation
\begin{equation}
R = U W^{-1} V^T
\end{equation}

\subsubsection{Obtaining depth}
This algorithm has been first described by Longuet-Higgins \cite{Longuet-Higgins87}. For brevity, we just summarize the most important equations.


\bibliographystyle{plain}
\bibliography{Report_2_Stefan_Moser}
\end{document}